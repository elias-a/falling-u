\documentclass{article}


\usepackage{fullpage}
\usepackage{graphicx,pgf}
\usepackage{amsmath}
\usepackage{amssymb}
\usepackage{amsthm}
\usepackage{array}


\usepackage{verbatim}


% \newcommand{\e}{\epsilon}
\newcommand{\ve}{\varepsilon}
\newcommand{\vp}{\varphi}
\newcommand{\ood}{\overline{\overline{\delta}}}
\newcommand{\og}{\overline{\lambda}(\gamma)}
\newcommand{\oog}{\overline{\overline{\lambda}}(\gamma)}
\newcommand{\T}{\mathbb{C}\slash \Lambda}
\newcommand{\Tn}{\mathbb{R}^n \slash \Lambda}
%\newcommand{\T3}{\mathbb{R}^3 \slash \Lambda}
\renewcommand{\theequation}{\arabic{section}.\arabic{equation}}
\newcommand{\re}{\operatorname{Re}}
\newcommand{\im}{\operatorname{Im}}
\newcommand{\e}{\operatorname{e}}
\newtheorem{lemma}{Lemma}[section]
\newtheorem{proposition}[lemma]{Proposition}
\newtheorem{corollary}[lemma]{Corollary}
\newtheorem{theorem}[lemma]{Theorem}
\newtheorem{definition}[lemma]{Definition}
\title{Dynamics of the falling U}
\author{Elias Alstead, Asa Alstead}

\usepackage{subcaption}
\usepackage{pgfplotstable}
\usepackage{array}  
\usepackage{csvsimple}
\usepackage{booktabs}

\newcolumntype{P}[1]{>{\centering\arraybackslash}p{#1}}

\usepackage[framed,numbered,autolinebreaks,useliterate]{mcode}

\begin{document}
\maketitle

\section{The problem}
\setcounter{equation}{0}

This study attempts to simulate the dynamics of a nontrivial extended object falling in a uniform gravitational field. Necessarily, the dynamics of the object in free fall, during impact with a fixed surface, and during rotation about a fixed point are investigated. Initially, this effort was undertaken to study the probability that an object lands on one of its particular sides. By computationally modeling the problem, we hope to create a program where assumed parameters can be varied with ease to examine the behavior of the object in different scenarios.

\section{The model}
\setcounter{equation}{0}

The natural starting point of our model is to define the dynamics of the center of mass while in free fall in a gravitational field. Say the center of mass of the object is moving with velocity $v$, and the extended object is rotating about the center of mass with angular velocity $\omega$. As an initial simplification, assume that the center of mass has velocity only in the y direction. Following Newton's laws, we can update the object's state during a discrete time step like
\begin{equation}
  r_{cm} \mathrel{+}= v \Delta t + \frac{1}{2} g {\Delta t}^2,
\end{equation}
and
\begin{equation}
    v_{cm} \mathrel{+}= g \Delta t.
\end{equation}
Furthermore, we can model the rotation of the object about an axis orthogonal to the plane in which the object's position is defined. Computationally, the angular position changes like
\begin{equation}
    \theta_{n + 1} = (\theta_{n} + \omega \Delta t) \pmod{2 \pi}
\end{equation}
where $\omega$ is the angular velocity.

\section{The U object}
\setcounter{equation}{0}

The U object has base length $b$ and side length $s$. It follows that the center of mass $c$ is
\begin{equation}
    c = \frac{s^2}{2 s + b}.
\end{equation}
The distance of the U from the ground is an important quantity needed in the simulation. We define a procedure that computes the distance from the ground $D$ for different values of $\theta$. First, we define an angle $\alpha$ that is restrained to the interval $[0,\pi]$. We relate $\alpha$ to the angular position by saying $\alpha = \theta$ when $\theta \in [0,\pi]$ and $\alpha = 2 \pi - \theta$ otherwise. Breaking the computation of $D$ into four scenarios, we have 
\[
D = 
\begin{cases}
    r_{cm} - c \\
    r_{cm} - \sqrt{\frac{b^2}{4} + c^2} (\tan^{-1}{\frac{b}{2 c}} - \alpha) & \text{if $\alpha \in (0,\frac{\pi}{2})$} \\
    r_{cm} - \frac{b}{2} & \text{if $\alpha = \frac{\pi}{2}$} \\
    r_{cm} - \sqrt{\frac{b^2}{4} + (s - c)^2} (\pi - \alpha - \tan^{-1}{\frac{b}{2 (s - c)}}) & \text{if $\alpha \in (\frac{\pi}{2},\pi)$} \\
    r_{cm} - s + c & \text{if $\alpha = \pi$}
\end{cases}
\]

\section{The simulation}
\setcounter{equation}{0}

\subsection{Propagate until contact with the ground}
The simulation can be broken into a number of smaller steps. The first is to consider the U object in free fall until it first makes contact with the ground. Using the formulation above, this step is trivial. Propagate $r_{cm}$, $v_{cm}$, and $\theta$ until $D = 0$.

\subsection{Bounce until $v = 0$}
Next, we take things a step further by propagating the dynamics past the initial contact. Let the impact be instantaneous and define a variable $\mu \in [0,1]$ called the surface factor that regulates momentum loss of the U upon impact with the ground. When the U object contacts the ground, update the velocity like
\begin{equation}
    v_{cm} = - v_{cm} \mu
\end{equation}
The velocity decreases and changes direction. We repeat this sequence of propagation and momentum loss until the velocity is sufficiently close to zero.

\subsection{Rotate upon impact}
The next iteration of the simulation involves dropping the assumption that the U object's impact with the ground is instantaneous. Define $\tau$ as the length of time the U is contact with the ground. While $v = 0$ during the duration $\tau$, gravity applies a torque to the object. We can derive the approximation of the torque by starting with
\begin{equation}
    \omega \vec{r}_{cm} = 
    \begin{bmatrix}
        \omega r_x \\
        \omega r_y
    \end{bmatrix}
\end{equation}
and considering the change in angular velocity during the time $\tau$. 
\begin{equation}
    \begin{bmatrix}
        \omega r_x \\
        \omega r_y
    \end{bmatrix}
    + 
    \begin{bmatrix}
        0 \\
        g \tau 
    \end{bmatrix}
    =
    \begin{bmatrix}
        w' r'_x \\
        w' r'_y
    \end{bmatrix}
\end{equation}
Using the fact that $r^2_x + r^2_y = r'^2_x + r'^2_y$, we have
\begin{equation}
    \omega r_x = w' r'_x
\end{equation}
so we determine the angular velocity after impact is
\begin{equation}
    \omega' = \frac{w r_y + g \tau}{r'_y}
\end{equation}
The goal now is to find representations of $r'_x$ and $r'_y$. Plugging the updated angular velocity back in and substituting $r'_y$ out, we get the relation
\begin{equation}
    \frac{\omega r_y + g \tau}{\sqrt{r^2_x + r^2_y - r'^2_x}} = \frac{\omega r_x}{r'_x}
\end{equation}
Solving for $r'_x$, we arrive at
\begin{equation}
    r'_x = \sqrt{\frac{\omega^2 r^2_x (r^2_x + r^2_y)}{\omega^2 r^2_x + (\omega r_y + g \tau)^2}}
\end{equation}
Completing the solution, we can write down $r'_y$ using $r'_y = \sqrt{r^2_x + r^2_y - r'^2_x}$.

Considering again the scenario from the previous section where we assumed instantaneous time of impact, we can reconcile our solution here with our previous model by letting $\tau \rightarrow 0$. The result is
\[ \lim_{\tau\to\infty} r'_x = \sqrt{\frac{\omega^2 r^2_x (r^2_x + r^2_y)}{\omega^2 r^2_x + (\omega r_y)^2}} = r_x \]
Similarly, $r'_y = r_y$ and $\omega' = \omega$ and we recover the previous result where the center of mass position and the angular velocity stayed constant during impact.

\end{document}